\documentclass[12pt,twoside]{article}
\usepackage{amsfonts,amsmath,amssymb,amsthm,color,arydshln}
\usepackage{epsfig}
\textheight=19.9cm \textwidth=14.8cm \topmargin=0.5cm
\oddsidemargin=1cm \evensidemargin=1cm

\everymath{\displaystyle}
\def\l{\left}
\def\r{\right}
\def\theequation{\arabic{section}.\arabic{equation}}
\newcommand{\bb}[1]{\begin{equation}\label{#1}}
\newcommand{\ee}{\end{equation}}
\newcommand{\bbb}{\begin{eqnarray}}
\newcommand{\eee}{\end{eqnarray}}
\newcommand{\bbbb}{\begin{eqnarray*}}
\newcommand{\eeee}{\end{eqnarray*}}
\newcommand{\nnn}{\nonumber}
\newcommand{\no}{\noindent}
\newcommand{\tT}{\intercal}
\def\TT{\mathbb{T}}
\def\RR{\mathbb{R}}
\def\CC{\mathbb{C}}
\def\DD{{\cal D}}
\def\OO{{\cal{O}}}
\def\R#1{$(\ref{#1})$}
\def\dia{\mbox{diamond-$\alpha$}}
\def\diast{{\diamondsuit_{\alpha}}}
\definecolor{green1}{rgb}{0.1,0.5,0.0}
\definecolor{blue1}{RGB}{51,78,125}
\definecolor{brown}{RGB}{220,90,49}
\definecolor{red1}{rgb}{0.8,0.1,0.0}
\definecolor{orange}{rgb}{1,0.5,0}
\definecolor{gray2}{rgb}{0.8,0.8,0.8}
\newcommand{\darkgreen}{\color{green1}}
\newcommand{\darkblue}{\color{blue1}}
\newcommand{\darkred}{\color{red1}}
\newcommand{\orange}{\color{orange}}
\newcommand{\blue}{\color{blue}}
\newcommand{\gray}{\color{gray2}}
\newcommand{\red}{\color{red}}
\newtheorem{Theorem}{Theorem}[section]
\newtheorem{Lemma}[Theorem]{Lemma}
\newtheorem{Corollary}[Theorem]{Corollary}
\newtheorem{Proposition}[Theorem]{Propoision}
\newtheorem{Definition}[Theorem]{Definition}
\newtheorem{remark}[Theorem]{Remark}
\usepackage[utf8]{inputenc}
\usepackage[english]{babel}
\usepackage{graphicx}
\graphicspath{ {images/} }

 
\newtheorem{theorem}{Theorem}[section]
\newtheorem{corollary}{Corollary}[theorem]
\newtheorem{lemma}[theorem]{Lemma}
\theoremstyle{definition}
\newtheorem{definition}[theorem]{Definition} % definition numbers are dependent on theorem numbers


\begin{document}
\begin{center}
{\Large\bf HW 10}
\end{center}
{\large\bf Chong Sun}
\begin{itemize}
\item (d) The linear discriminant scores of the point $x_0 = [3.5 ~~1.75]'$ for groups Iris setosa, Iris versicolor and Iris virginica are given respectively by
\bbbb
d_1 &=& 51.56,\\
d_2 &=& 23.77,\\
d_3 &=& -1.12.
\eeee
Thus we classify the point into group Iris setosa.

\item (f) We have the APER$=0.047. $

To find out the $\hat{E}(\text{AER}),$ we use stratified 10-fold classification method described as follows.
\begin{itemize}
\item Radomly partitioned the original data set into 10 equal sized subsamples.
\item Of the 10 subsamples, a single subsample is retained as the validation data for testing the model, and the remaining 9 subsamples are used as training data.
\item The above process is then repeated 10 times (the folds), with each of the 10 subsamples used exactly once as the validation data.
\item The 10 results from the folds can then be averaged to produce a single estimation.
\item  We then repeat the above cross-validation procedure 100 times, yielding 100 random partitions of the original sample. The 100 results are again averaged to produce the final estimate of error rate.
\end{itemize}

We have $\hat{E}(\text{AER})=0.053$ as our final result.
\end{itemize}

\end{document}






